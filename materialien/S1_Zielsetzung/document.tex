\documentclass[a4paper,oneside]{scrarticle}

\usepackage[left=3cm,right=3cm,top=2cm,bottom=2.25cm]{geometry}
\usepackage[ngerman]{babel}
\usepackage{amsmath}
\usepackage{amsfonts}
\usepackage{amssymb}
\usepackage{mathtools}
\usepackage{graphicx}
%\graphicspath{ {./images/} }
\addto\captionsngerman{\renewcommand{\figurename}{Fig.}}


\begin{document}
	\begin{flushleft}
		Bach Nguyen - HTWK Leipzig - INB (Draft)
	\end{flushleft}
	\begin{center}
		\begin{LARGE}
			\textbf{Problem of the Month (February 2007)}
		\end{LARGE}
	\end{center}
	\section*{Problem description}
	In February 2007, we investigated positions of 2 kinds of chess pieces, each of which the same number of the other piece but none of their own. Is there a position where each bishop attacks 2 knights and each knight attacks 3 bishops? We abbreviate this as (B=2, N=3). Here are all the unsolved cases: (B=2, K=4), (B=4, Q=3), (K=2, Q=5), (N=3, R=3), (N=4, R=2).
	\section*{(K=2, Q=5)}
	In this case, we will try to disproof of any possible finite solutions.\\
	k:= King (piece), q:= Queen (piece)\\
	$\mathbb{K}$ set of all King pieces, $\mathbb{Q}$ set of all Queen pieces\\
	\\
	We will be using the following notation:\\
	\begin{figure}[h]
		\centering
		\includegraphics[width=0.3\linewidth]{notation}
		\caption{notation}
		\label{fig:notation}
	\end{figure}
	\\
	The position of a piece will be a tuple (n,m), where n is the row and m the column. Ex. $k_3 = (2,2), k_2 = (n,3), k_1 = (3,m)$.\\
	Furthermore we define a two projection functions to give us the piece's row and column\\
	$\pi_r:\mathbb{K}\cup\mathbb{Q}\rightarrow\mathbb{N}$\\
	$\pi_r(p) = \pi_r((x,y)) = x$\\
	$\pi_c:\mathbb{K}\cup\mathbb{Q}\rightarrow\mathbb{N}$\\
	$\pi_c(p) = \pi_r((x,y)) = y$\\
	\\
	An open King for a Queen in South, means there are at least one space and no other pieces between these two.\\
	\\
	Directions are named after the cardinal direction.\\
	\begin{figure}[h]
		\centering
		\includegraphics[width=0.2\linewidth]{carnialdirection}
		\caption{cardinal direction}
		\label{fig:carnialdirection}
	\end{figure}
	
		
	\pagebreak
	
	\section*{Proof by exhaustion}
	
	\section{definition $k_r,k_b$}
	Let us assume there is a possible solution. Then there has to be one most outer piece to the right. Without loss of generality, due to axial and rotation symmetry in all 4 directions, we can then infer all other symmetries.\\
	\\
	If a solution exists, there has to be at least 5 Kings and 2 Queens. Now that there are more than 2 Kings, then there has to be at least one King that is farthest to the right. We will call this column m, which means
	\begin{flalign}
		\exists!m\in \mathbb{N}\ \forall k\in\mathbb{K}:m\geq\pi_c(k)\land m\in\pi_c(\mathbb{K})
	\end{flalign}
	From all k there is but one piece that is most south $k_r = (x,m)$ with $x\in\{1,2,\dots,n\}$\\
	Similarly there has to be at least one King in the most south row $n$ that is farthest to the east, which we denote the position $k_b = (n,y)$ with $y\in\{1,2,\dots,m\}$ with $n,m\in\mathbb{N}$
	\begin{flalign}
		\exists!k_r\forall k\in\{p|\pi_c(p)=m\land p\in\mathbb{K}\}: \pi_r(k_r)\geq\pi_r(k)\\
		\exists!k_b\forall k\in\{p|\pi_r(p)=n\land p\in\mathbb{K}\}: \pi_c(k_b)\geq\pi_r(k)
	\end{flalign}
	\\
	
	\begin{figure}[h]
		\centering
		\includegraphics[width=0.3\linewidth]{example_edge}
		\caption{example edge}
		\label{edge case}
	\end{figure}
	
	\section{$k_r \neq k_b$}
	
	Firstly, let us rule out the case where $k_r = k_b$. This means there is in fact just one piece and $\forall k\in\mathbb{K}: \pi_c(k)\leq\pi_c(k_r) \land \pi_r(k)\leq\pi_r(k_r)$\\
	If this configuration is part of a solution, then let us consider where the queen pieces can be placed in order to satisfy our condition.\\
	\begin{figure}[h]
		\centering
		\includegraphics[width=.2\linewidth]{kbkr_equal}
		\caption{$k_r = k_b$ queen positioning}
		\label{fig:kbkrequal}
	\end{figure}
	\\
	If we take a look at $q_0$, then we can see that in order to threaten 5 Kings in total, only 3 Kings can be placed in NW,W and SW direction. Which also means that at least 2 Kings have to be placed in the remaining directions, which violates our premise for $k_r$ to be the most outer piece. Similarly, $q_1,q_2,q_3,q_4$ would also violate it.\\
	This means for $k_r = k_b$, the 2 Queens have to be placed within the NW,W and SW direction. But this is also impossible because then the Queens would threaten themselves which violates our problem definition.
	
	\section{Queens of $k_r$}
	
	Continuing the reasoning using $k_r$ as the most south of the most east king pieces of a possible finite solution, we can first rule out the case where one of the queens are placed in any of the east neighbor tiles.\\
	\begin{figure*}[h]
		\centering
		\includegraphics[width=0.4\linewidth]{QueenKingExpansion}
		\caption{Queen Threats}
		\label{fig:queenkingexpansion}
	\end{figure*}\\
	If there is a $q_r\in\mathbb{Q}$ east to $k_r$, then there has to be at least two other Kings placed either in the same column or even further to the east than $q_r$. We will call this "forced king expansion to the east". This violates our premise of $k_r$ and will be deemed impossible.
	\\
	\begin{figure*}[h]
		\centering
		\includegraphics[width=0.9\linewidth]{krQueentiles}
		\caption{$k_r$ Queen neighbors all configurations King threatening 2 Queens}
		\label{fig:krqueentiles}
	\end{figure*}\\
	Out of all configurations only 3 remain: 6,7 and 9.\\
	\\
	What happens for 6 and 7 when a Queen is placed South of $k_r$?\\
	Again we shall not expand to the East. Only one possible configuration for this piece to threaten 5 Kings remain by placing another King to the south. This also, violates our premise (2). Therefore no such solution would exist.\\
	\\
	The only pattern left is 9 (Fig. 6) and now we know that in a finite solution, the Queens threatened by $k_r$ have to be placed N and SW.
	
	\section{Queens of $k_b$}
	Due to symmetry, we can infer the only possible shape of Queens around $k_b:$ pattern 1 in Fig. 6.\\
	\begin{figure}[h]
		\centering
		\includegraphics[width=0.4\linewidth]{patternkrkb}
		\caption{pattern $k_r$ and $k_b$}
		\label{fig:patternkrkb}
	\end{figure}
	
	\section{Standalone King expansion}
	Given any Queen piece, because there are only 8 directions and 5 of them threatening a King it holds that there has to be at least one pair of Kings in opposite direction.\\
	With this, for any standalone King $k_a\in\mathbb{K}$ we can deduce the following.\\
	If there is a Queen in either S, SE or E, then they would expect at least another King further south or east compared to this King.
	\\
	\begin{figure}[h]
		\centering
		\includegraphics[width=0.5\linewidth]{standaloneKingQueen}
		\caption{standalone $k_a$ for Queen in SE,S,E}
		\label{fig:standalonekingqueen}
	\end{figure}
	\\
	In order to not expand a King from $k_a$ in direction S, SE, E, only configurations 1,9,10 remain (Fig.6).
	\\
	\begin{figure}[h]
		\centering
		\includegraphics[width=0.5\linewidth]{standaloneKingQueenremain}
		\caption{remaining options no S,SE,E}
		\label{fig:standalonekingqueenremain}
	\end{figure}
	
	
	

	\section{King expansion of $k_r$}
	Now let us continue by examining the Queen $q_r$ SW to $k_r$. Let's try to find a solution where we can avoid a King expansion in direction S,SE or E. Immediately we can see, E and SE should be free of Kings (2). While threatening Kings to N, NW, W and obviously also $k_r$ only 2 directions remain S, SW. Because of Q=5 there has to be a King in at least one of them.
	\\
	Only 4 cases remain: 
	\\
	a) Let there be an attached King $k_1$ S to $q_r$.
	\\
	b) Let there be an open King $k_2$ S to $q_r$.
	\\
	c) Let there be an open King $k_3$ SW to $q_r$.
	\\
	d) Let there be an attached King $k_4$ SW to $q_r$.
	\\
	Case a):\\
	Not expanding to SE only S and SW remain. 
	Then there is a forced expansion to SW either attached or open which would lead back to our 4 cases for a new Queen, just further to the South. Therefore, we can safely ignore this case for now.
	\\
	Case b):\\
	If a Queen is open to the South, then the corresponding King must be open to the North. This King again needs 2 Queens. In case of any Eastern neighbor, another King expansion to S, SE or E would be forced.
	The only shape that still holds (Fig.6) would be shape 6 with Queens to NW and S. Now another King is forced to the South and with one more Queen to SW, this brings us back to our 4 cases with again a queen S to $k_2$ but only more to the south. Just like in a) we can also ignore this case because it repeats the pattern only further to the South.
	\\
	\begin{figure}[h]
		\centering
		\includegraphics[width=0.9\linewidth]{4caseqr}
		\caption{4 cases for $q_r$}
		\label{fig:4caseqr}
	\end{figure}
	\\
	Case c):\\
	Similarly, in order not to expand E, SE or S we need to avoid placing Queens in said direction. Being open to NE, leaves us with the only shape exactly like $k_r$ which also leads us back to our 4 cases. Because this King $k_3$ has to be at least in the diagonal SW to $q_r$ we can ignore this case. But actually this case is impossible because if this is the only shape for $k_3$ and with it being open to $q_r$, we can see that both queens North to $k_r$ and $k_b$ would threaten each other. With no possible King being placed in between because for that King it would be impossible to place two Queens in only N, NW and W.
	\\
	Case d):\\
	Threatening any King in S direction would lead us back to case a) or b). Threatening any King in SW direction would lead us back to c) or d). Unlike in c) because having a Queen NE provides one more option (shape 1) in Fig.~\ref{fig:standalonekingqueenremain} . And again this was a King expansion to SW. Here we should also rule out the option that we could place the queen West to $k_4$ because that would be threatened by the Queen North to $k_r$. Even if d) is appended to previous a) or b) there will still always be that Queen.
	\\
	\\
	And after examining all 4 cases, while trying to avoid any expansion to S, SE or E, we found out that it is impossible to avoid expansion to either S or SW for any King forcibly expanded through $k_r$. We could also rule out any expansion to NE, E, SE or S like discussed in section 5.\\
	\\
	It follows that in a finite solution with $k_r$ expanding it has to eventually meet the lowest King piece. Which means there either is no solution or it has to be in the same row as $k_b$.
	
	\section{King expansion of $k_b$}
	Analogous to $k_r$ we can say that $k_b$ must also expand at least diagonally to NE or further south.
	
	\section{forced if $k_r$ then $k_b$ at least at diagonal}
	In the previous sections we did see that from $k_r$ a King expansion is forced at least directly to the South but at least diagonally to $k_r$. This means $k_b$ has to be at least in that diagonal its row or further to the East. 
	
	\section{if $k_b$ past diagonal $k_b$ not possible}
	On the other hand, due to the symmetrical property $k_b$ also has to expand at least diagonally or further North and finally meet $k_r$.
	
	\section{diagonal not possible}
	Unfortunately, this deems impossible. Because if $k_b$ is further East to the diagonal SW of $k_r$, then $k_b$ forces $k_r$ to be at least in the diagonal NE to $k_b$. In the column we assumed $k_r$ before, now shows us $k_r$ being further to the South than before. So we can conclude that $k_r$ has to be diagonally placed to $k_b$ and otherwise. The only solution where this would be possible, is when $k_r$ is using the pattern c) or d) in Fig.\ref{fig:4caseqr}, where c is already impossible.\\
	Because with a Queen on top N to $k_r$ and a Queen W to $k_b$ and the alternating sequence of Kings on our diagonal, no further pieces can be placed between these two Queens. But this can also not be a solution because Queens are not allow to threaten each other just like in case d).\\
	\\
	Therefore, no solution to (K=2,Q=5) can exist. \qquad\qquad\qquad\qquad\qquad\qquad$\square$
	


\end{document}
